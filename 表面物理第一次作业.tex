\documentclass[reqno,a4paper,12pt]{amsart}

\usepackage{amsmath,amssymb,amsthm,geometry,xcolor,soul,graphicx}
\usepackage{titlesec}
\usepackage{enumerate}
\usepackage{lipsum}
\usepackage{listings}
\RequirePackage[most]{tcolorbox}
\usepackage{braket}
\allowdisplaybreaks[4] %align公式跨页
\usepackage{xeCJK}
\setCJKmainfont{Kai}
\geometry{left=0.7in, right=0.7in, top=1in, bottom=1in}

\renewcommand{\baselinestretch}{1.3}

\title{表面物理作业}
\author{董建宇 ~~ 202328000807038}

\begin{document}

\maketitle
\titleformat{\section}[hang]{\small}{\thesection}{0.8em}{}{}
\titleformat{\subsection}[hang]{\small}{\thesubsection}{0.8em}{}{}

\begin{enumerate}
\item 证明有14种Bravais晶体。
\begin{tcolorbox}[breakable, colback = black!5!white, colframe = black]
对于七大晶系的加心操作,部分操作会使得晶体点阵与已有点阵相同,并不构成新的Bravais点阵。

举例如下:

对于单斜晶体,添加面心后形成面心单斜晶体,该晶体点阵等价于三斜晶体点阵。如下图:

\begin{center}
	\includegraphics[scale = 0.18]{face_center_T.jpeg}
\end{center}

对于单斜晶体增加体心,仍等价于三斜晶体点阵,如下图:

\begin{center}
	\includegraphics[scale = 0.23]{body_center_T.jpeg}
\end{center}

对于正方晶体增加面心,该晶体点阵等价于三斜晶体点阵。如下图:

\begin{center}
	\includegraphics[scale = 0.23]{face_center_ZF.jpeg}
\end{center}

对于正方晶体增加底心,该晶体点阵等价于三斜晶体点阵。如下图:

\begin{center}
	\includegraphics[scale = 0.23]{up_center_ZF.jpeg}
\end{center}

同样地,三方晶体点阵增加面心后与三方晶体点阵等价;三方晶体点阵增加底心后与单斜晶体点阵等价;三方晶体点阵增加体心心后与三方晶体点阵等价。

\end{tcolorbox}

\item 证明$d_{hkl} = \frac{2\pi}{\vert \vec{G}_{hkl} \vert}$
\begin{tcolorbox}[breakable, colback = black!5!white, colframe = black]
考虑
\[
	\vec{G}_{hkl} = h\vec{b}_1 + k\vec{b}_2 + l\vec{b}_3, ~~ \vec{a}_i \cdot \vec{b}_j = \delta_{ij}.
\]
则对于晶面上格点向量$\vec{R} = n_1\vec{a}_1 + n_2\vec{a}_2 + n_3 \vec{a}_3$有:
\[
	\vec{G}_{hkl} \cdot \vec{R} = (hn_1+ln_2+kn_3)2\pi = 2m\pi.
\]
其中$m$为整数。则相邻两个晶面间距离$d$满足:
\[
	\vert \vec{G}_{hkl} \vert d = 2\pi.
\]
即有:
\[
	d_{hkl} = \frac{2\pi}{\vert \vec{G}_{hkl} \vert}
\]
\end{tcolorbox}
\end{enumerate}
\end{document}