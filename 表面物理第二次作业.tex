\documentclass[reqno,a4paper,12pt]{amsart}

\usepackage{amsmath,amssymb,amsthm,geometry,xcolor,soul,graphicx}
\usepackage{titlesec}
\usepackage{enumerate}
\usepackage{lipsum}
\usepackage{listings}
\usepackage{multirow}
%\RequirePackage[most]{tcolorbox}
\usepackage{braket}
\allowdisplaybreaks[4] %align公式跨页
%\usepackage{xeCJK}
%\setCJKmainfont[AutoFakeBold = true]{Kai}
\geometry{left=0.7in, right=0.7in, top=1in, bottom=1in}

\renewcommand{\baselinestretch}{1.3}

\title{Homework of Surface Physics}
\author{Jianyu Dong \ \ \ \ 202328000807038}

\begin{document}

\maketitle

\begin{enumerate}[1.]

\item For $Si$ crystal, find the $2\theta$ value and relative intensities for the first $5$ existing diffraction peaks in powder XRD, assume X-ray wavelength of $0.5$\AA.

\begin{proof}
	We could get the structure of Si is FCC with two basis at $(000)$ and $(\frac{1}{4}\frac{1}{4}\frac{1}{4})$ from $Springer \ Materials$ with lattice parameters $a = 5.43$\AA. There are $8$ atoms in each unit cell and the relative positions of atoms are 
	\[
		\vec{r}_0 = (0,0,0); \ \vec{r}_1 = (1/4, 1/4, 1/4); \ \vec{r}_2 = (1/2, 1/2, 0); \ \vec{r}_3 = (3/4, 3/4, 1/4); 
	\]
	\[
		\vec{r}_4 = (1/2, 0, 1/2); \ \vec{r}_5 = (3/4, 1/4, 3/4); \ \vec{r}_6 = (0, 1/2, 1/2); \ \vec{r}_7 = (1/4, 3/4, 3/4).
	\]
	
	The vector basis are 
	\[
		\vec{a}_1 = a(1,0,0); \ \vec{a}_2 = a(0,1,0); \ \vec{a}_3 = a(0,0,1).
	\]
	
	Then we could determine the volume of an unit cell
	\[
		V = \vec{a}_1 \cdot \vec{a}_2 \times \vec{a}_3 = a^3.
	\]
	
	The reciprocal basis are 
	\[
		\vec{b}_1 = \frac{2\pi}{a}(1,0,0); \ \vec{b}_2 = \frac{2\pi}{a}(0,1,0); \ \vec{b}_3 = \frac{2\pi}{a}(0,0,1).
	\]
	
	Then we could determine the structure factor as below 
	\begin{align*}
		F =& \sum_{j=0}^7 f_j e^{-i\vec{G}_{hkl}\cdot\vec{r}_j} = f \sum_{j=0}^7 e^{-i\vec{G}_{hkl}\cdot\vec{r}_j} \\
		=& f\left( 1+e^{-i\frac{\pi}{2}(h+k+l)} \right) \left( 1 + e^{-i\pi(h+k)} + e^{-i\pi(h+l)} + e^{-i\pi(k+l)} \right).
	\end{align*}

	The conditions for $F=0$ are that $h,k,l$ are all even or all odd or $\frac{h+k+l}{2}$ is odd.
	
	The interplanar crystal spacing with Miller indices $(hkl)$ is
	\[
		d_{hkl} = \frac{2\pi}{\vert \vec{G}_{hkl} \vert} = \frac{a}{\sqrt{h^2+k^2+l^2}}.
	\]
	
	Then we could determine the five least interplanar crystal spacing and multiplicity factor $M(h,k,l)$ are 
	
	\setlength{\tabcolsep}{5mm}{
	\begin{table}[h!]
	\centering
	\begin{tabular}{|c|c|c|c|c|c|}
		\hline
		$hkl$ & 111 & 220 & 311 & 400 & 331 \\
		\hline
		$d_{hkl}$/\AA & 3.14 & 1.92 & 1.64 & 1.36 & 1.25 \\
		\hline
		$M(hkl)$ & 8 & 12 & 24 & 6 & 24 \\
		\hline
	\end{tabular}
	\end{table} }
	
	Utilizing the Bragg's diffraction formula 
	\[
		2d\sin\theta = n\lambda,
	\]
	
	we could determine the diffraction angles for different $n$ and different indices as below 
	
	\setlength{\tabcolsep}{5mm}{
	\begin{table}[h]
	\centering
	\begin{tabular}{|c|c|c|c|c|c|}
		\hline
		$hkl$ & 111 & 220 & 311 & 400 & 331 \\
		\hline
		$d_{hkl}$/\AA & 3.14 & 1.92 & 1.64 & 1.36 & 1.25 \\
		\hline
		$\theta = \arcsin \frac{\lambda}{2d}$ & 4.57 & 7.48 & 8.77 & 10.59 & 11.54 \\
		\hline
		$2\theta$ & 9.14 & 14.96 & 17.54 & 21.19 & 23.07 \\
		\hline
%		$\theta = \arcsin \frac{2\lambda}{2d}$ & 9.16 & 15.09 & & & \\
%		\hline 
%		$2\theta$ & 18.33 & 30.19 & & & \\
%		\hline 
%		$\theta = \arcsin \frac{3\lambda}{2d}$ & 13.82 & & & & \\
%		\hline
%		$2\theta$ & 27.64 & & & & \\
%		\hline
	\end{tabular}
	\end{table} }
	
	Then we could get the five smallest angles and the number of equivalent indices for each diffraction theta $m$ are 
	
	\setlength{\tabcolsep}{5mm}{
	\begin{table}[h]
	\centering
	\begin{tabular}{|c|c|c|c|c|c|}
		\hline
		$\theta$ & 4.57 & 7.48 & 8.77 & 10.59 & 11.54 \\
		\hline
		$2\theta$ & 9.14 & 14.96 & 17.54 & 21.19 & 23.07 \\
		\hline
		$M(hkl)$ & 8 & 12 & 24 & 6 & 24 \\
		\hline
	\end{tabular}
	\end{table} }
	
	Utilizing the formula 
	\[
		I_{hkl} \propto M(h,k,l) \times \vert F(h,k,l) \vert^2 \times \frac{1+\cos^2(2\theta)}{\sin^2\theta\cos\theta},
	\] 
	
	the relative intensity for each $d_{hkl}$ with corresponding value of $\theta$ is 
	
	\setlength{\tabcolsep}{5mm}{
	\begin{table}[h]
	\centering
	\begin{tabular}{|c|c|c|c|c|c|}
		\hline
		$hkl$ & 111 & 220 & 311 & 400 & 331 \\
		\hline
		$\theta$ & 4.57 & 7.48 & 8.77 & 10.59 & 11.54 \\
		\hline
		$2\theta$ & 9.14 & 14.96 & 17.54 & 21.19 & 23.07 \\
		\hline
		$M(hkl)$ & 8 & 12 & 24 & 6 & 24 \\
		\hline
		$F(h,k,l)$ & $4(1+i)f$ & $8f$ & $4(1-i)f$ & $8f$ & $4(1+i)f$ \\
		\hline
		$\vert F \vert^2/\vert f \vert^2$ & 32 & 64 & 32 & 64 & 32 \\
		\hline
		$I_{hkl}/I_{max}$ & 0.90 & 1 & 0.72 & 0.24 & 0.41 \\
		\hline
	\end{tabular}
	\end{table} }
\end{proof}

\medskip

\item An element, BCC or FCC, shows diffraction peaks at $2\theta$ of $40, 58, 73, 86.8, 100.4$ and $114.7$ (with X-ray wavelength of $1.54$\AA). Determine: (a) Crystal structure? (b) Lattice constant? (c) What is the element?

\begin{proof}
\begin{enumerate}[(a)]

\item According to the Bragg's diffraction formula 
\[
	2d\sin\theta_i = n_i \lambda,
\]
we could get the table as below.

\setlength{\tabcolsep}{5mm}{
\begin{table}[h]
\centering
\begin{tabular}{|c|c|c|c|c|c|c|}
	\hline
	$2\theta$ & 40 & 58 & 73 & 86.8 & 100.4 & 114.7 \\ \hline
	$\theta$ & 20 & 29 & 36.5 & 43.4 & 50.2 & 57.35 \\ \hline
	$d=\frac{\lambda}{2\sin\theta}$ & 2.25 & 1.59 & 1.29 & 1.12 & 1.00 & 0.91 \\ \hline
	$(d_{max}/d)^2$ & 1.00 & 2.00 & 3.04 & 4.04 & 5.06 & 6.06 \\ 
	\hline
\end{tabular}
\end{table} }
\newpage
Since $N = h^2+k^2+l^2 = 1$ is not allowed for neither $FCC$ nor $BCC$, we need to consider the integral multiple of $(d_{max}/d)^2$. 

If we consider $2(d_{max}/d)^2$, we could get the table as below 

\setlength{\tabcolsep}{5mm}{
\begin{table}[h]
\centering
\begin{tabular}{|c|c|c|c|c|c|c|}
	\hline
	$2(d_{max}/d)^2$ & 2 & 4 & 6 & 8 & 10 & 12 \\
	\hline
	$(hkl)$ & (110) & (200) & (211) & (220) & (310) & (222) \\
	\hline
\end{tabular}
\end{table} }

It is clearly that $h+k+l$ is even, which shows that the crystal structure could be BCC.

If we consider $3(d_{max}/d)^2$, we could get the table as below 

\setlength{\tabcolsep}{5mm}{
\begin{table}[h]
\centering
\begin{tabular}{|c|c|c|c|c|c|c|}
	\hline
	$3(d_{max}/d)^2$ & 3 & 6 & 9 & 12 & 15 & 18 \\
	\hline
	$(hkl)$ & (111) & (211) & (300)/(221) & (222) & none & (330) \\
	\hline
\end{tabular}
\end{table} }

Since $(211)$ is forbidden for FCC, the crystal structure could not be FCC. 

Thus, the crystal structure is BCC.

\item The lattice constant $a$ satisfy 
\[
	d_{max} = \frac{a}{\sqrt{1^2+1^2+0^2}} = \frac{a}{\sqrt{2}}.
\]

So the lattice constant is $a = 3.18$\AA.

\item Compared with the database, we could get the element could be W (a=3.16\AA) or Mo (a=3.15\AA)
\end{enumerate}

\end{proof}

\medskip

\end{enumerate}

\end{document}